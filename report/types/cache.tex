\subsection{缓存探测}

当用于访问一个网页,由于用户再次访问同一个网页的概率较大,浏览器会将某些资源,比如图像、脚本、HTML 代码,缓存在用户的机器上。当用户再次访问用一个网页,浏览器不必从服务器再次下载,而可以更快速地从本地存储读取,从而加快了网页加载的速度。缓存探测漏洞基于检测某个资源是否被缓存,从而攻击者可以判断被攻击者是否曾经访问有个网页。

此类漏洞有多种实现方法。一种简单的方法是使用时序攻击技巧。这种攻击来自于缓存本质的用途,即若某个资源被缓存了,则它的访问时间较短,反之亦然。


Related:

\begin{itemize}
    \item \url{https://www.cs.jhu.edu/~fabian/courses/CS600.424/course_papers/webtiming.pdf} - OLD, from 2000, gives experimental results, also discusses DNS and cookie cache
    \item \url{https://terjanq.github.io/Bug-Bounty/Google/cache-attack-06jd2d2mz2r0/index.html} - error based cache attack on Google products
    \item \url{http://sirdarckcat.blogspot.com/2019/03/http-cache-cross-site-leaks.html} - addresses some basic defense strategies
    \item \url{https://web.archive.org/web/20200614162731/http://u.cs.biu.ac.il/~herzbea/security/15-01-XSSearch.pdf} - improvements on timing attacks, uses statistics, amplification, and DaC algs, not specific to cache probing
    \item \url{https://link.springer.com/content/pdf/10.1007/978-3-319-18467-8.pdf} pdf page 110, parallized cache probing
\end{itemize}

\subsubsection{错误事件}

\begin{enumerate}
    \item 使被缓存的资源无效:
    \begin{itemize}
        \item 使用 \code{cache:'reload'} 发出请求,在收到响应之前使用 \code{AbortController.abort()} 终止
        \item 使用 \code{cache:'reload'} 以及 overlong referer header
        \item A POST request with a fetch no-cors
        \item 将请求失败的 Content-Type, Accept, Accept-Language 等等请求头,必须针对由一个网站
    \end{itemize}
    \item 发出请求,使得某一个资源被缓存
    \item 再对同一个资源发出请求,但需要将服务器拒绝此请求(比如使用 overlong referer header)。若此资源在第二步被缓存了,则此请求会成功,否则抛出错误
\end{enumerate}

\subsubsection{CORS error on Origin Reflection misconfiguration}

若响应包含 Access-Control-Allow-Orign (ACAO),发出请求的来源以及被请求的资源一起被缓存在本地。若 \code{attacker.com} 访问此资源:

\begin{itemize}
    \item 若此资源未被缓存,此资源以及 \code{Access-Control-Allow-Orign (ACAO): attacker.com} 将被缓存
    \item 若此资源被缓存,由于 \code{attacker.com} 与以缓存的 \code{target.com} 不匹配,会产生 \code{CORS} 错误,从而可以判定此资源被缓存过
\end{itemize}

容易避免:在资源上设置 \code{Access-Control-Allow-Origin: *}

\subsubsection{Fetch with AbortController}

见 错误事件

\subsubsection{防范措施}

\begin{itemize}
    \item 通过设置 \code{Cache-Control: no-store} 禁用缓存,非常简单、高效地防止此类攻击,并被大多数浏览器支持,但对网页的加载速度有负面的影响
    \item 在资源的 URL 中加随机记号,比如 \code{users/john.jpg} 变成 \code{users/john.jpg?cache_buster=<RANDOM_TOKEN>}。依然可以使用缓存机制,从而不会影响到加载速度,但是必须由每个网站的管理人员实现,而不是由浏览器实现,可以保护所有网站的防范措施
    \item Fetch metadata: 可以让服务器判定请求来自于相同还是不同的来源。比如如果资源的 URL 为 \code{cdn.example.com/image.png} 并且设置了 \code{Vary: Sec-Fetch-Site}(SFS),则:
    \begin{table}[h!]
        \centering
        \begin{tabular}{cc}
            请求来源 & SFS \\\midrule
            \code{example.com} & \code{same-site} \\
            \code{cdn.example.com} & \code{same-origin} \\
            \code{evil.com} & \code{cross-site} \\
        \end{tabular}
    \end{table}
    依然可以使用缓存机制但是弊端包括:
    \begin{itemize}
        \item 并非所有浏览器支持 fetch metadata
        \item 跨站点资源无法被保护
        \item 资源若被第三方访问,也无法被保护
    \end{itemize}
\end{itemize}