\section{基于 Service Worker 与 Performance API 的 Chromium 漏洞}

\subsection{概念验证}

\href{https://blog.lbherrera.me/}{Luan Herrera} 在 2019 年提出了基于 service worker 以及 Performance API,影响到 Chromium 的攻击 \cite{herrera}。Service worker \url{https://developer.mozilla.org/en-US/docs/Web/API/Service_Worker_API} 

Performance API(性能接口)让开发者精确地测量网页在用户的设备上的性能 \cite{papi}。此攻击

攻击过程如下:

\begin{enumerate}
    \item 安装拦截 range header(字节范围头)为 \code{bytes=0-} 的请求的 Service Worker。
    \item 使用 \code{audio} 或 \code{video} 元素向目标资源发出请求,其请求中的字节范围头为 \code{bytes=0-}。
    \item Service Worker 拦截以上的请求,并返回任意内容,长度为 \code{n} 的响应。Chromium 会再发出类似的请求,类以区别是 \code{bytes=0-} 变成了 \code{bytes=n-}。分两种情况:
    \begin{enumerate}
        \item 目标资源的大小小于 \code{n},则请求失败,服务器返回 \code{416} 状态码,此事件不产生 \code{PerformanceEntry}
        \item 目标资源的大小大于 \code{n},则请求成功,服务器返回 \code{206} 状态码,此事件产生 \code{PerformanceEntry}
    \end{enumerate}
    \item 使用 \code{performance.getEntries().length} 可以得知当前的请求是否产生了 \code{PerformanceEntry},从而可以判断目标资源的大小是否小于 \code{n}。
\end{enumerate}

通过改变 \code{n},比如用二分查找,可以较快地判定目标资源的大小。比如,Herrera 对 Google 的错误追踪系统的 XS-Search 攻击就可以使用此漏洞。

(测试包含 service worker 的网站后,为了注销 service worker, 可能需要重启浏览器程序)。

\subsection{修复方法}

由于概念验证中检测资源大小基于成功与失败创建 \code{PerformanceEntry} 的行为的差异 \url{https://chromium.googlesource.com/chromium/src/+/5e556dd80e03b7a217e10990d71be25d07e1ece7}

\begin{quote}
    Currently we don't report performance entries with failing status codes.
    From the spec's perspective, reporting aborts is a MAY, but failing
    status code responses should not be considered aborts. [1]
    Chromium is the only engine which doesn't report those entries.
    This CL fixes that to report them similarly to successful status codes.
\end{quote}

此修复的具体实现比较简单。源代码的版本库将 \url{https://chromium-review.googlesource.com/c/chromium/src/+/1796544}

\begin{lstlisting}
if (resource->GetResponse().IsHTTP() &&
    resource->GetResponse().HttpStatusCode() < 400)
\end{lstlisting}

改成 

\begin{lstlisting}
if (resource->GetResponse().IsHTTP())
\end{lstlisting}


即任何响应,无论状态码表示成功或失败,都会产生一个 \code{PerformanceEntry}。


通过运行 Luan 的概念验证,可以确认此修复成功地防止此攻击。

\subsection{扩展}

Luan 的概念验证

一个发现是只有 \code{audio} 以及 \code{video} 元素在 URL 中加入 \code{size} 参数时,浏览器才会

不稳定