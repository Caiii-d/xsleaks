\section{形式模型}

\citeauthor{modelbase} \cite{modelbase} 定义了跨站点泄露漏洞的形式模型。根据他们的定义,跨站点泄露漏洞是输出一个比特 $b'$ 的函数 $xsl$:
$$b'=xsl(sdr, i, t)$$

其输入为:

\begin{itemize}
    \item $sdr$:依赖于状态的资源(state-dependent resource),而它是二元组 $(url, (s, d))$,其中 $(s, d)\in\{(s_0, d_0), (s_1, d_1)\}$:
    \begin{itemize}
        \item $url$:目标资源的 URL,可以包括若干个查询参数,比如 \url{https://www.google.com/search?q=xsleaks}。
        \item $S=\{s_0, s_1\}$:网站的两个状态的集合。
        \item $D=\{d_0, d_1\}$:网站的行为的差异,依赖于 $s_0$ 和 $s_1$。
    \end{itemize}
    \item $i\in I$:包含方法(inclusion method),即如何将攻击网站向 $sdr$ 发出请求
    \item $t\in T$:泄露技术(leak technique),即如何判别目标网站上的差异
\end{itemize}

$(s,d)$ 的定义比较抽象因为对于不同的浏览器,不同的 URL,不同的一对状态,或不同的 $i$ 可以展现出不同的差异。比如,若 $s_0$ 是用户未登录的状态,$s_1$ 是用户登录的状态,若使用缓存探测 (章节 \ref{sec:cache}),$d_0$ 可以是用户发布的帖子未缓存,而 $d_1$ 是用户发布的帖子以缓存,可以设 $i$ 为时序攻击判别 $d_0$ 以及 $d_1$。或者,若登录和未登录状态下的网页有不同的 \code{frame} 个数,可以用计帧法(章节 \ref{sec:frame})判别两个状态。

其次,虽然一个网页可以有多个状态,但可以简化到二元、是否问题。比如如果一个邮件的标题可以有无限多种标题,但通过判别“第一个字母是否是 ‘a’”、“第一个字母是否是 ‘b’”、……、“第二个字母是否是 ‘a’”等等逐渐获得完整的标题。