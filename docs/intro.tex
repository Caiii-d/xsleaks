\section{引言}

\subsection{基础概念}

来源(Origin)\cite{origin}: Web 内容的 URL 的方案 (协议),主机 (域名) 和端口。两个对象是同源的当且仅当以上三个元素都相等。\\

旁道攻击(Side-channel attack):通过观察系统运行过程中的物理属性从而获取信息,而非暴力破解或运用算法的理论弱点。\\

跨站点泄露漏洞(Cross-site leaks, XSLeaks)\cite{xsleaks}:一类存在与浏览器中的旁道攻击,让一个站点获取另一个站点的某些信息。\\


例子:\code{evil.com} 想得知用户在 \code{google.com} 上的搜索的关键字。\code{evil.com} 可以发出请求 \code{https://google.com/search?q=a}, \code{https://google.com/search?q=b} …… 并测量从发出请求和接受到响应之间的时间间隔。若请求中发出的关键字是用户被曾经访问过,响应时间与未访问过的关键字响应相比会较短,因为本地会缓存响应中一部分的数据。\\

\subsection{形式化建模}

\subsubsection{基础模型\cite{modelbase}}

跨站点泄露漏洞是输出一个比特 $b'$ 的函数 $xsl$

$$b'=xsl(sdr, i, t)$$

其输入为:

\begin{itemize}
    \item $sdr$:依赖于状态的资源(state-dependent resource),而它是二元组 $(url, (s, d))$,其中 $(s, d)\in\{(s_0, d_0), (s_1, d_1)\}$:
    \begin{itemize}
        \item $url$:目标资源的 URL
        \item $S=\{s_0, s_1\}$:网站的两个状态的集合
        \item $D=\{d_0, d_1\}$:网站的行为的差异,依赖于 $s_0$ 和 $s_1$
    \end{itemize}
    \item $i\in I$:包合技术,即如何从攻击网站向 $sdr$ 发出请求
    \item $t\in T$:泄露技术,即如何观察目标网站上的差异
\end{itemize}

在以上的例子中:

\begin{itemize}
    \item $url=$ \code{https://google.com/search?q=[query]}
    \item $S=\{$ \code{query} 未被查询, \code{query} 已被查询 $\}$
    \item $D=\{$ 时间间隔较长, 时间间隔较短 $\}$
    \item $i$ 可以使用多种方式,比如将 $url$ 嵌入到攻击网站中的 \code{iframe}
    \item $t$ 为计时攻击
\end{itemize}

\subsubsection{扩展模型 \cite{modelext}}

\citeauthor{modelext} 用 \citeauthor{modelbase} 的模型为基础,作出了扩展:


\subsection{COSI}

跨源状态推断(Cross-Origin State Inference)攻击\cite{cosi}:

\begin{itemize}
    \item 考虑两个网站
    \begin{itemize}
        \item 攻击网站:用于发出跨站点请求,是攻击者(部分)控制的网站
        \item 目标网站:用户在此网站上有不同的状态,不被攻击者控制的网站
    \end{itemize}
    \item 攻击网站中含有依赖于状态的网址(state-dependent URL, SD-URL),比如一个只有用户登录后才能访问的网页
    \item 被包含的 SD-URL 使用户的浏览器发出跨源请求,但同源策略防止攻击网站直接阅读响应
    \item 攻击者可以通过跨站点泄露漏洞间接地读取响应
\end{itemize}